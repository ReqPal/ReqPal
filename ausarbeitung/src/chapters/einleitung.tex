\chapter{Einleitung}

\authoredsection{Laura}{Hintergrund}

Im Bereich der Wirtschaftsinformatik ist die Evaluierung von
Standardanwendungs-Software und das Bewerten von funktionalen Anforderungen von entscheidender Bedeutung.
Standardanwendungssoftware unterscheidet sich von Individualsoftware dadurch, dass sie nicht für einen spezifischen Anwendungsfall entwickelt wird.\footcite[Vgl.][S. 3]{Teich2008}{}{} Stattdessen ist sie darauf ausgelegt, von verschiedenen Nutzern in unterschiedlichen Kontexten verwendet zu werden.
Dementsprechend ist es eine wichtige Aufgabe, die Software auf die vom Anwender benötigten Anforderungen und Kriterien zu überprüfen. Kriterien sind zum Beispiel die Funktionalitäten der Software, der Preis, ihre Anpassbarkeit und ihre Komplexität.

Diese Aufgaben können für Studierende der Wirtschaftsinformatik ein Bestandteil ihres Berufsalltags  werden. Daher ist es von entscheidender Bedeutung, dass sie während ihres Studiums mit diesen Herausforderungen vertraut gemacht werden und anhand von praktischen Übungen Erfahrungen sammeln können.

Das Projekt WiLMo (Wirtschaftsinformatik Lehr- und Lern-Module) beschäftigt sich unter anderem mit der Erstellung und Bereitstellung öffentlich zugänglicher Lern-Materialien zu diesem Thema.\footcite[Vgl.][]{wilmo}{}{}
Ein spezieller Fokus der Fachhochschule Dortmund im Rahmen dieses Projekts liegt auf der Entwicklung einer Übungsumgebung für das Thema \glqq Auswahl und Einführung betrieblicher Anwendungssysteme\grqq
. Diese Arbeit konzentriert sich genau auf diesen Aspekt.
Die Digitalisierung von Übungen mittels einer Online-Lernplattform eröffnet neue Wege zur Gestaltung von Übungsabläufen und Bearbeitungsprozessen.
Insbesondere die Interaktionsmöglichkeiten spielen hier eine bedeutsame Rolle.
Verschiedene Interaktionsmöglichkeiten ermöglichen es Lernenden, tiefer in die Lernumgebung einzutauchen und ihre eigene Lernmotivation zu steigern.\footcite[Vgl.][S.638]{BolognaDigital}{}{}
Durch das Fehlen direkter sozialer Interaktionen, die bei Einzelarbeiten typischerweise weniger präsent sind, wird es umso wichtiger, andere Formen der Interaktion zu fördern.\footcite[Vgl.][S.638]{BolognaDigital}{}{}

Aus diesem Grund legt die Arbeit neben der Entwicklung der Übungsumgebung einen starken Fokus auf die Erstellung von Aufgaben, die die Interaktion und Motivation der Lernenden gezielt fördern und unterstützen.

\authoredsection{Fabian}{Zielsetzung}

Das primäre Ziel unserer Lernplattform besteht darin, einen interaktiven Kontext für Studierende der Wirtschaftsinformatik zu schaffen, in dem sie die Fähigkeit erlangen und trainieren können, verschiedene Standardanwendungsprodukte anhand funktionaler Anforderungen zu analysieren und zu vergleichen. Ein wesentliches Lernziel ist die Vermittlung der Fähigkeit, Anforderungen effektiv zu bewerten und die Eignung verschiedener Softwareprodukte kritisch zu beurteilen. Dies umfasst das Sammeln und Auswerten zusätzlicher Produktinformationen, um fundierte Entscheidungen treffen zu können.

Zu diesem Zweck sollen Lehrende in der Lage sein, erstellte Anforderungskataloge auf der Plattform zu hinterlegen und daraus interaktive Übungen zu entwickeln, die den Studierenden zur Verfügung gestellt werden. Die Aufgaben sollen ansprechend und benutzerfreundlich gestaltet sein, um eine hohe Lernmotivation und aktive Teilnahme zu fördern.

Weiterhin ist es geplant, dass die Möglichkeit bestehen soll, die Applikation innerhalb des Projektes WILMO weiterzuentwickeln. Aus diesem Grund legen wir von Anfang an großen Wert darauf, die Plattform so zu gestalten, dass sie gut wartbar ist. Dies soll durch den Einsatz moderner Entwicklungs- und Kollaborationstools wie Projektmanagement-Boards und Ticketsysteme erreicht werden. Ziel ist es, ein robustes Dokumentations- und Supportsystem zu schaffen, das die Plattform nicht nur für aktuelle, sondern auch für zukünftige Entwicklungsschritte zugänglich und handhabbar macht.

Ein weiterer wichtiger Schwerpunkt liegt auf der Auswahl der Implementierungstechnologien. Wichtig ist uns, dass diese Technologien öffentlich verfügbar und idealerweise Open Source sind, um eine breite Basis für Unterstützung und Weiterentwicklung zu gewährleisten. Zudem ist es entscheidend, dass die ausgewählten Technologien auch in den kommenden Jahren aktiv unterstützt werden, um die Langlebigkeit und Nachhaltigkeit der Plattform sicherzustellen.

\authoredsection{Fabian}{Zusammengefasste Vorgehensweise}

Vor der eigentlichen Implementierung unserer Anwendung wurde eine umfangreiche Analyse des Kontextes durchgeführt. Im ersten Abschnitt der Analyse wurde basierend auf dem bereitgestellten Material ein didaktisches Konzept entwickelt. Der Fokus lag auf der Untersuchung der Form und Struktur, in der Anforderungskataloge erstellt werden können, sowie auf der Entwicklung von Aufgaben, die auf diesen Katalogen basieren. Es wurde analysiert, wie diese Aufgaben strukturiert sein könnten und welche Interaktionsmöglichkeiten sich daraus ergeben könnten.

In der darauffolgenden Phase wurde auf Grundlage der erarbeiteten Ergebnisse eine detaillierte Anforderungsanalyse für das Projekt durchgeführt. In dieser Analyse wurden spezifische Aspekte wie der Anwendungskontext, der Systemkontext, das Domänenmodell und die Stakeholder der Applikation eingehend betrachtet und definiert. Anschließend wurden auf Basis dieser Schritte User Stories entwickelt, die die Bedürfnisse und Anforderungen der Endnutzer widerspiegeln. Aus diesen User Stories wurde schließlich eine Anforderungsspezifikation abgeleitet, die als Leitfaden für die weitere Projektentwicklung dient.

Für die Entwicklung der Lernplattform wurden zunächst bestehende Lösungen analysiert und grob evaluiert, um sicherzustellen, dass mit vorhandenen Technologien kein gleichwertiges oder zumindest ein dem nahekommendes Ergebnis erzielt werden kann, welches die zuvor ermittelten Anforderungen erfüllt.

Dafür wurde das an der FH Dortmund genutzte Learning Management System ILIAS und das Konkurrenzprodukt Moodle betrachtet und mit dem erarbeiteten Anforderungskatalog verglichen. Es stellte sich heraus, dass sich mit den von den LMS gegebenen Mitteln kein maßgeschneidertes Produkt ableiten ließ, ohne einen erheblichen Aufwand im Schreiben von Erweiterungen zu dem entsprechenden System.

Basierend auf diesen Erkenntnissen wurde entschieden, eine eigene webbasierte Plattform zu entwickeln. Dafür wurde zunächst mit einer Entwurfs- und Designphase begonnen, in der GUI-Prototypen entworfen, grundlegende Technologien verglichen und ausgewählt sowie ein Plan für das Projektmanagement und Deployment erstellt wurden. Zudem wurde ein Datenbankmodell erarbeitet.

Basierend auf diesen Schritten haben wir einen High-Fidelity-Prototypen der Lernplattform entwickelt. Dieser wurde dann mit Dozenten und Studierenden evaluiert und erprobt. Daraus abgeleitete Wünsche und Hinweise wurden in das Planungsboard übernommen und kritische Aspekte auch direkt umgesetzt.

Abschließend wird ein Ausblick für die Zukunft der Lernplattform gegeben.