\chapter{Analyse}

\section{Lernprozess}
\subsection{Anforderungskatalog}
\subsubsection{Struktur}
\subsubsection{Aufgaben}
\subsection{Interaktion und Motivation}
- auf Theorien verweisen z.B. im Bezug auf das Punktesystem: Kompetenzerleben, Testing Effect, Motivation steigern und Lernergebnisse verbessern (Quellen z.B. Serious Games and Edutainment, Bologna Digital)

\section{Anforderungsanalyse}
\subsection{Einleitung}
\subsection{Systemkontext}
\subsection{Anwendungskontext}
\subsection{Domänenmodell}
\subsection{Stakeholder}
\subsection{User Stories}

\section{Anforderungsspezifikation}
\subsection{Einleitung}
\subsection{Funktionale Anforderungen}
\subsection{Nicht funktionale Anforderungen}
\subsection{Anhang}
\subsection{Index}

\section{Bestehende Lernplattformen}
\authoredsubsection{Laura}{Einleitung}
Basierend auf den zuvor ermittelten Anforderungen ergibt sich nun die Möglichkeit, bestehende Lernumgebungen einer kritischen Analyse und einem Vergleich zu unterziehen. Dieser Prozess dient dazu, die Relevanz und den Mehrwert einer potenziellen neuen Lernplattform zu untersuchen.

Hierfür werden zwei bekannte Plattformen betrachtet: Moodle und Ilias.
Die Bewertung dieser Lernumgebungen erfolgt durch die eingehende Analyse ihrer Funktionen sowie durch das praktische Testen der Umgebung und der Erstellung eigener Aufgaben.

\authoredsubsection{Laura}{Moodle}
Moodle ist eine Online-Lernplattform, die Lehrenden die Möglichkeit bietet, diverse Formen von Online-Unterricht und Schulungen zu gestalten.\footcite{moodle}
Als Open-Source-Lernmanagement-System (LMS) ist es kostenfrei verfügbar und ermöglicht eine breite Anpassung an individuellen Vorstellungen.

Lehrende können mithilfe von Moodle Kurse erstellen und diese mit verschiedenen Elementen wie Lektionen, Übungen und Workshops gestalten. Die Plattform bietet eine Vielzahl an Funktionen, darunter zum Beispiel, die Bereitstellung von vielfältigen Lerninhalten, die Schaffung von Kommunikationsmöglichkeiten sowie die Interaktion der Lernenden untereinander.

Die Kursgestaltung erfolgt über ein Formular. Es können Bilder und Videos hinzugefügt werden. Innerhalb der Kurse können dann Lektionen und Übungsaufgaben hinzugefügt werden.
Diese Übungen umfassen vielfältige Quizformate mit unterschiedlichen Vorlagen. Es gibt zum Beispiel Multiple Choice, True oder False oder der Zuordnung von Inhalten (Drag and Drop) zu Bildern. Die Zusammenarbeit mit anderen Lernenden wird mithilfe von Workshops ermöglicht. Außerdem können eigene Dateien hochgeladen werden, die dann in die Übung miteinbezogen werden können. 

Die Lektionen selbst können unterschiedlich konzipiert werden. Einerseits mit dem Fokus zum Erlernen eines neuen Themas oder mit dem Schwerpunkt auf das Treffen von Entscheidungen mit Szenarien.

Darüber hinaus können Aktivitäten wie Umfragen, Entscheidungsfindungen und die Überprüfung des Verständnisses der Lernenden integriert werden. 

Im Profil der Lernenden ist eine Übersicht über persönliche Daten und Noten zu finden. 
Weiterhin existiert ein Kalender mit Terminen und der Lernende kann allgemeine Benachrichtigungen erhalten.
Lehrende haben zudem die Möglichkeit, Aufgaben gezielt an Lernende zuzuweisen und diese anschließend zu bewerten.

Insgesamt bietet Moodle für Bildungseinrichtungen eine solide Grundlage, um online Lehrmaterialien zur Verfügung zu stellen und vollständige Kurse zu entwickeln. 
Durch die Open-Source-Natur wird die Zugänglichkeit erweitert. 
Wenn spezifische Funktionen fehlen oder Anpassungen in den Kursen erforderlich sind, können Plug-Ins eine Lösung bieten. Diese zusätzlichen Erweiterungen ermöglichen es, fehlende Elemente zu ergänzen oder die Kurse gemäß individueller Anforderungen anzupassen.

Jedoch zeigt sich bei der Erstellung von Aufgaben mit Anforderungskatalogen für die Evaluierung von Standardsoftware eine gewisse Komplexität. Das Hochladen von kompletten Katalogen ist nicht direkt möglich und somit könnte das Einbinden von Anforderungen in Aufgaben erschwert werden.
Trotz interaktiver Übungsaufgaben, fehlen einige motivierende Elemente, wie die Integration von einem Punktesystem oder Simulationen mit einem Unternehmen, um das Engagement zu fördern.
Diese können jedoch durch die Entwicklung individueller Plug-Ins mithilfe der Skriptsprache PHP ergänzt werden.
Die Entwicklung und Integration von Anpassungen mittels PHP und Moodle können einen beträchtlichen Aufwand erfordern. Es stellt sich daher die Frage, ob das Verhältnis von diesem Aufwand zu den erwarteten Vorteilen und Nutzen solcher Anpassungen gerechtfertigt ist.

\authoredsubsection{Fabian}{ILIAS}
\authoredsubsection{}{Fazit}