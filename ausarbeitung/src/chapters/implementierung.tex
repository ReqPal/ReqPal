\chapter{Implementierung}

\section{Anforderungskataloge}
\authoredsubsection{Fabian}{Hochladen eines Kataloges mit Edgefunctions}

\authoredsubsection{Laura}{Anzeigen von hochgeladenen Katalogen}
Nachdem die Kataloge erfolgreich hochgeladen wurden, wie im Abschnitt zuvor beschrieben, sind Lehrende dazu autorisiert, diese zu betrachten, zu löschen oder in den Lektionen miteinzubeziehen. 
Diese Anforderungskataloge gehören zu den wichtigsten Bestandteilen der Lernplattform und spielen eine bedeutsame Komponente in den Lektionen. 
Sie bestehen aus vielen, verschiedenen Daten, darunter die Anforderungen selbst und die Softwareprodukte, die dazu betrachtet wurden. Für jedes Produkt gibt es jeweils eine Qualifizierung und einen Kommentar pro Anforderung. Diese Komplexität der Kataloge weist eine gewisse Herausforderung beim Aufbau und Anzeigen dieser Daten auf. 
Zunächst gilt es, die Daten abzufragen. 

\inputminted{TypeScript}{assets/code/catalogs/fetch_catalogs_store.ts}

- store, service, supabase und zusammenfügen der daten erläutern

- Komplexität der Daten und Tabellen
- Herausforderung Tabellen darzustellen, ggf auf Problematik mit noch nicht fertigen UI Komponenten von vuetify eingehen und der Entscheidung für eine vorübergehende Alternative (-> genutzte Technologien) anstatt selber zu programmieren
- Anzeige der wichtigsten Elemente, Vgl zu vorheriger Darstellung und möglicher Verbesserungen (z.B. Direkte Ansicht von Produkten und deren Qualifizierungen)
- Funktionen, Zugriffsschutz/Rechte

\section{Lektionen}
\authoredsubsection{Fabian}{Aufgabenerstellung}

\authoredsubsection{Laura}{Bearbeitung}
\subsubsection{Punkteberechnung mit Datenbankfunktionen}
- ggf Diagramm zum Ablauf des Abschließens der Lektion, der Punkteberechnung, Speicherung und der Anzeige der Ergebnisse (Prozess einfacher zu verstehen)
- Ausschnitt von den Datenbankfunktionen und wie die Berechnung und Prüfung erfolgt für versch. Fragetypen (techn. Umsetzung)
- Verhindern von falschen Ergebnissen oder mehrfacher Punktezuschreibung (Datenintegrität, ehrliche Bearbeitung durch Lernende)
- Feedback je nach erreichter Punktzahl (Einblick in Motivationsstärkung)
- Schwierigkeiten/Herausforderungen bei der Punkteberechnung

\authoredsection{Fabian}{Sicherheit}
\subsection{Policies}
\inputminted{Sql}{assets/code/sql/policy_catalogs.sql}
\subsection{Authentifizierung}
\subsection{Rollen und Rechte}
\subsection{Autorisierung über Middlewares}